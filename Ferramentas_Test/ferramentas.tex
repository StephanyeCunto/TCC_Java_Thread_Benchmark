\documentclass[a4paper,10pt]{article}
\usepackage[utf8]{inputenc}
\usepackage{geometry}
\usepackage{listings}
\usepackage{xcolor}

\geometry{margin=2cm}

\definecolor{codegray}{rgb}{0.95,0.95,0.95}
\definecolor{keyword}{rgb}{0.36,0.54,0.66}
\definecolor{comment}{rgb}{0.5,0.5,0.5}

\lstdefinestyle{myCodeStyle}{
    backgroundcolor=\color{codegray},
    commentstyle=\color{comment}\itshape,
    keywordstyle=\color{keyword}\bfseries,
    basicstyle=\ttfamily,
    breaklines=true,
    frame=single,
    language=bash
}

% Ab style igual ao myCodeStyle para padronização
\lstdefinestyle{abstyle}{
    backgroundcolor=\color{codegray},
    commentstyle=\color{comment}\itshape,
    keywordstyle=\color{keyword}\bfseries,
    basicstyle=\ttfamily,
    breaklines=true,
    frame=single,
    language=bash
}

\begin{document}

\section{Vegeta}
\begin{itemize}
    \item \textbf{Como usar:}
    \begin{lstlisting}[style=myCodeStyle]
echo "GET http://localhost:8080/threads/virtual" | vegeta attack -duration=1s -rate=19000 | vegeta report
    \end{lstlisting}
    \item Limite aproximado: 19000 requisições por segundo
\end{itemize}

\section{JMeter}
\begin{itemize}
    \item Criar arquivo \texttt{.jmx}
    \item Limite: número de threads depende da capacidade do PC
    \item Em minha máquina: aproximadamente 1500 a 2300 requisições por segundo
\end{itemize}

\section{Gatling}
\begin{itemize}
    \item Não consegui instalar
\end{itemize}

\section{k6}
\begin{itemize}
    \item Criar arquivo \texttt{.js}
    \item Limite aproximado: 5000 requisições por segundo
\end{itemize}

\section{ApacheBench (ab)}

\begin{lstlisting}[style=abstyle]
ab [opções] [URL]
\end{lstlisting}

Exemplo:
\begin{lstlisting}[style=abstyle]
ab -n 100 -c 10 http://localhost:8080/teste
\end{lstlisting}

Onde:
\begin{itemize}
    \item \texttt{-n 100} → número total de requisições
    \item \texttt{-c 10} → número de requisições concorrentes
\end{itemize}

\section{Principais opções do ab}

\subsection{Requisições e concorrência}
\begin{lstlisting}[style=abstyle]
-n <numero>  # Total de requisições
-c <numero>  # Conexões simultâneas
\end{lstlisting}

\subsection{Cabeçalhos HTTP}
\begin{lstlisting}[style=abstyle]
-H "Header: valor"     # Adiciona cabeçalho customizado
-A usuario:senha       # Autenticação básica HTTP
\end{lstlisting}

\subsection{Tipo de requisição}
\begin{lstlisting}[style=abstyle]
-p arquivo       # Envia dados POST a partir de arquivo
-T tipo          # Tipo de conteúdo (ex: application/json)
-u arquivo       # Envia dados PUT a partir de arquivo
\end{lstlisting}

\subsection{Cookies}
\begin{lstlisting}[style=abstyle]
-C "nome=valor"  # Envia cookie na requisição
\end{lstlisting}

\subsection{TLS / HTTPS}
\begin{lstlisting}[style=abstyle]
-s segundos     # Timeout para cada requisição
-Z cipherlist   # Cipher SSL customizado
-X host:porta   # Proxy HTTP
-v nivel        # Verbosidade (0 a 4)
\end{lstlisting}

\subsection{Logs e relatórios}
\begin{lstlisting}[style=abstyle]
-e arquivo.csv   # Salva resultados detalhados em CSV
-g arquivo.dat   # Salva dados para plot no gnuplot
-r               # Não aborta em erro HTTP
\end{lstlisting}

\subsection{Testes avançados}
\begin{lstlisting}[style=abstyle]
-k           # Habilita Keep-Alive
-v nivel     # Mostra detalhes de cabeçalhos e respostas
-q           # Suprime saída de progresso
\end{lstlisting}

\section{Exemplos completos}

\subsection{Teste GET simples}
\begin{lstlisting}[style=abstyle]
ab -n 500 -c 50 http://localhost:8080/teste
\end{lstlisting}

\subsection{Teste POST com JSON}
\begin{lstlisting}[style=abstyle]
ab -n 200 -c 20 -p dados.json -T application/json http://localhost:8080/api
\end{lstlisting}

\subsection{Teste com Keep-Alive e headers personalizados}
\begin{lstlisting}[style=abstyle]
ab -n 1000 -c 100 -k -H "Authorization: Bearer TOKEN" http://localhost:8080/teste
\end{lstlisting}

\subsection{Salvar resultados em CSV}
\begin{lstlisting}[style=abstyle]
ab -n 1000 -c 100 -e resultados.csv http://localhost:8080/teste
\end{lstlisting}


\end{document}
