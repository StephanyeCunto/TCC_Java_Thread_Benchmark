\documentclass[12pt, openright, oneside, a4paper, brazil]{abntex2}

\usepackage{fontspec}
\usepackage{graphicx}			
\usepackage{xcolor}	
\usepackage{microtype} 			
\usepackage{setspace}
\usepackage{indentfirst}		
\usepackage{float}
\usepackage{caption}
\usepackage{tabularx}  
\usepackage{pdflscape}  
\usepackage{geometry}

\geometry{left=2cm, right=2cm, top=2cm, bottom=2cm}

\newfloat{quadro}{htbp}{loq}[section]
\floatname{quadro}{Quadro}
\captionsetup[quadro]{font=small, labelfont=bf, labelsep=period}

\begin{document}

\begin{landscape}
    \begin{quadro}[H]
        \centering
        \caption{Trabalhos relacionados} 
        \fontsize{10}{12}\selectfont
        
        \begin{tabularx}{\linewidth}{|p{7cm}|p{7cm}|p{3cm}|X|}
            \hline
            \textbf{Nome} & \textbf{Métricas} & \textbf{Benchmark} & \textbf{Resumo}  \\ \hline

            Comparison of Concurrency Technologies in Java & 
            
            Latência (mínimo, máximo, médio, 50º, 90º, 95º, 99º percentil), taxa de transferência, taxa de resposta do servidor (para taxa de sucesso), bytes de entrada e bytes de saída. & 
            
            Não  disponível &  
            
            Analisou latência, taxa de transferência, taxa de resposta do servidor e bytes de entrada e saída de um teste de E/S (simula uma operação de bloqueio de E/S simples com operações computacionais mínimas ou nulas), teste de computação (contém um cálculo com complexidade de tempo ao quadrado que visa carregar os diferentes aplicativos), teste misto - computação (realiza multiplicação d  uma matriz 200x200, seguida por uma operação de suspensão de 25 ms) e teste misto - E/S (realiza multiplicação de uma matriz 150x150, seguida por uma operação de suspensão de 100 ms).\\ \hline


            
            Avaliação dos mecanismos de concorrência na API do Java 8 & 
            
            Tempo de execução e consumo de memória. & 
            
            Não disponível &  
            
            Comparou o tempo de execução e consumo de memória dos algoritmos quicksort, mergesort, pidgeonholesort e JavaSort(método do próprio java) com Single Thread, Threads, ExecutorService e Fork/Join em três computadores com os processadores: T4300, i5-4260U e i7-3610QM.\\ \hline



            Uma análise comparativa entre threads e green threads no Java & 
            
            Tempo de execução. & 
            
            Utilizou a biblioteca \href{https://github.com/openjdk/jmh}{org.openjdk.jmh} & 
            
            Mediu o tempo de instanciamento, inicialização, sincronização (\textit{join}) e troca de contexto (\textit{yield}) de 100.000 (cem mil) threads em uma máquina AMD® Ryzen 7 3700u, 20GB de memória RAM e executando Ubuntu 22.04.03 LTS.\\ \hline



            Benchmarking the Performance of Java Virtual Threads in High-Throughput Workloads & Taxa de transferência, latência, uso de memória e utilização da CPU & 
            
            Não disponível & 
            
            Analisou taxa de transferência, latência, uso de memória e utilização da CPU de threads tradicionais (Executors.newFixedThreadPool()) e virtuais (Executors.newVirtualThreadPerTaskExecutor()) em cenários CPU-bound (cálculo de números primos) e I/O-bound (simulação de bloqueio entre duas aplicações Java com requisições HTTP) em um ambiente no AWS EC2.\\ \hline
        \end{tabularx}
    \end{quadro}
\end{landscape}

\end{document}
