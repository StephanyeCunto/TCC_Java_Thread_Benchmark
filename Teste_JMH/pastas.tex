\documentclass[12pt,a4paper]{article}
\usepackage{fontspec}
\usepackage{geometry}
\usepackage{longtable}
\usepackage{array}
\usepackage{graphicx}
\usepackage[table]{xcolor}

\geometry{margin=2cm}

\begin{document}

\begin{center}
\Large \textbf{Estrutura de Pastas do Repositório JMH}
\end{center}

\vspace{1em}

\begin{longtable}{
    |>{\raggedright\arraybackslash}p{4cm}|>{\raggedright\arraybackslash}p{11cm}|} \hline
    \textbf{Pasta/Módulo} & \textbf{Descrição} \\ \hline
    \endfirsthead \hline
    \textbf{Pasta/Módulo} & \textbf{Descrição} \\ \hline
    \endhead

    \cellcolor{green!40} .github/workflows & Workflows de CI/CD do GitHub para testes e builds automáticos. \\ \hline
    \cellcolor{green!40} .jcheck & Regras de verificação de qualidade e estilo de código do OpenJDK. \\ \hline
    \cellcolor{green!40} jmh-ant-sample & Exemplo de uso do JMH com Ant como sistema de build. \\ \hline
    \cellcolor{green!40} jmh-archetypes & Modelos de projeto Maven para criar novos projetos de benchmark JMH. \\ \hline
    \cellcolor{red!40} jmh-core-benchmarks & Benchmarks base ou de exemplo da infraestrutura core do JMH. \\ \hline
    \cellcolor{red!40} jmh-core-ct & Testes de compatibilidade ou “contract tests” da core do JMH. \\ \hline
    \cellcolor{red!40} jmh-core-it & Testes de integração (integration tests) da parte core do JMH. \\ \hline
    \cellcolor{green!40} jmh-core & Módulo principal do JMH, contendo classes e runners de benchmarks. \\ \hline
    \cellcolor{green!40} jmh-generator-annprocess & Processamento de anotações (annotation processor) para gerar código de benchmarks. \\ \hline
    \cellcolor{green!40} jmh-generator-asm & Geração de código de benchmarks via ASM (manipulação de bytecode). \\ \hline
    \cellcolor{green!40} jmh-generator-bytecode & Geração e manipulação de bytecode adicional para benchmarks. \\ \hline
    \cellcolor{green!40} jmh-generator-reflection & Geração de código via reflexão para benchmarks. \\ \hline
    \cellcolor{red!40} jmh-samples & Exemplos de uso do JMH para aprendizado e referência. \\ \hline
    \cellcolor{green!40} src/license & Contém arquivos de licenciamento do projeto. \\ \hline
    \cellcolor{green!40} THIRD-PARTY & Lista de bibliotecas de terceiros usadas no projeto e suas licenças. \\ \hline
    \cellcolor{green!40} pom.xml & Arquivo Maven principal do projeto, gerenciando dependências e módulos. \\ \hline
    \cellcolor{green!40} README.md & Documentação principal do repositório. \\ \hline
    \cellcolor{green!40} LICENSE & Arquivo da licença GPL-2.0 do projeto. \\ \hline
    \cellcolor{green!40} .gitignore & Arquivo para ignorar arquivos/diretórios no repositório Git. \\ \hline
\end{longtable}

\end{document}
