\documentclass[10pt,a4paper]{article}

\usepackage{fontspec} % XeLaTeX ou LuaLaTeX
\usepackage{setspace}
\usepackage{geometry}
\usepackage{titlesec}
\usepackage[table,xcdraw]{xcolor} % Para cores nas tabelas
\usepackage{booktabs} % Para tabelas mais elegantes
\usepackage{caption}

\geometry{margin=2.5cm}
\setstretch{1.3}

% Fonte principal

% Cores para títulos
\definecolor{sectioncolor}{RGB}{0, 102, 204}
\definecolor{subsectioncolor}{RGB}{0, 51, 102}

\titleformat{\section}{\normalfont\Large\bfseries\color{sectioncolor}}{\thesection}{1em}{}
\titleformat{\subsection}{\normalfont\large\bfseries\color{subsectioncolor}}{\thesubsection}{1em}{}

\title{\textbf{Descrição do Experimento: Teste de I/O com Atraso Fixo de 100 ms}}
\author{}
\date{}

\begin{document}

\maketitle

\section*{Introdução}

Este experimento utiliza um atraso fixo de 100 ms para simular operações de I/O.
As especificidades do aquecimento foram projetadas para executar cada método mais de
10.000 vezes, garantindo uma margem de segurança adequada para estabilização do ambiente.

\section{Fase de Aquecimento}

\begin{enumerate}
    \item Executar 60 segundos com 300 solicitações por segundo.
    \item Realizar coleta manual de lixo.
    \item Manter o sistema em modo de espera por 20 segundos.
    \item Repetir três vezes com taxas ajustadas para cada método.
\end{enumerate}

As três taxas subsequentes de aquecimento são definidas como 70\% da taxa máxima suportada por cada método.

\subsection{Resumo da Fase de Aquecimento}

\begin{table}[h!]
\centering
\caption{Parâmetros do aquecimento}
\begin{tabular}{@{}llll@{}}
\toprule
\rowcolor[HTML]{D9EAF7} 
Etapa & Duração & Taxa & Observações \\ \midrule
1 & 60 s & 300 req/s & Aquecimento inicial \\
2 & - & - & Coleta manual de lixo \\
3 & 20 s & - & Espera entre etapas \\
4 & - & 70\% taxa máxima & Repetir 3x por método \\ \bottomrule
\end{tabular}
\end{table}

\section{Experimentos com Carga Constante}

\textbf{Procedimento para cada ataque:}

\begin{enumerate}
    \item Enviar solicitações HTTP com taxa \texttt{RATE} durante 10 segundos e salvar os resultados.
    \item Operar em modo de repouso por 60 segundos.
    \item Executar coleta manual de lixo.
    \item Aguardar 20 segundos.
    \item Aumentar a taxa \texttt{RATE} em 25/50 e repetir o processo.
\end{enumerate}

\subsection{Resumo dos Experimentos}

\begin{table}[h!]
\centering
\caption{Etapas dos Experimentos com Carga Constante}
\begin{tabular}{@{}llll@{}}
\toprule
\rowcolor[HTML]{D9EAF7} 
Etapa & Duração & Taxa & Observações \\ \midrule
1 & 10 s & RATE & Coleta de métricas HTTP \\
2 & 60 s & - & Modo repouso \\
3 & - & - & Coleta manual de lixo \\
4 & 20 s & - & Pausa antes da próxima carga \\
5 & - & RATE+25/50 & Repetição ajustando taxa \\ \bottomrule
\end{tabular}
\end{table}

\section{Métricas Coletadas}

\subsection{Cliente}

A máquina cliente, utilizando o \textbf{Vegeta}, coleta as seguintes métricas:

\begin{table}[h!]
\centering
\caption{Métricas coletadas pelo cliente}
\begin{tabular}{@{}ll@{}}
\toprule
\rowcolor[HTML]{D9EAF7} 
Métrica & Descrição \\ \midrule
Latência & mínima, máxima, média, percentis 50, 90, 95, 99 \\
Taxa de transferência & Volume de dados processados por segundo \\
Taxa de amostragem & Frequência de coleta de amostras \\
Respostas do servidor & Taxa de sucesso (\%) \\
Bytes de entrada/saída & Quantidade de dados transmitidos \\ \bottomrule
\end{tabular}
\end{table}

\subsection{Servidor}

O servidor coleta as seguintes métricas:

\begin{table}[h!]
\centering
\caption{Métricas coletadas pelo servidor}
\begin{tabular}{@{}ll@{}}
\toprule
\rowcolor[HTML]{D9EAF7} 
Métrica & Descrição \\ \midrule
CPU & Utilização da CPU (\%) \\
Threads & Número de threads ativas e iniciadas \\
Heap & Heap total e heap utilizado \\ \bottomrule
\end{tabular}
\end{table}

\end{document}
