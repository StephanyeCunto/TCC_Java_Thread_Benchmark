\documentclass[12pt,a4paper]{article}
\usepackage{fontspec} 
\usepackage{geometry} 
\usepackage{fancyhdr} 
\usepackage{setspace} 
\usepackage{hyperref} 
\usepackage{parskip} 
\geometry{top=2.5cm, bottom=2.5cm, left=2.5cm, right=2.5cm}
\setmainfont{Times New Roman}

\pagestyle{fancy}
\fancyhf{}
\fancyfoot[C]{\thepage}
\renewcommand{\headrulewidth}{0pt}
\renewcommand{\footrulewidth}{0pt}

\begin{document}

  \section*{Página 29}

  3.  Experimentos  de  Aumento  Gradual  de  Carga:  Nos  experimentos  de  aumento  gradual  de  carga,  uma  máquina cliente  enviará  solicitações  simultâneas  a  um  servidor.  O  nível  de  concorrência  é  determinado  pelas  solicitações  por  segundo  e  aumentará  linearmente  até que  ocorra  a  saturação  de  recursos  no  servidor.  Durante  esses  testes,  as  métricas  do  cliente  e  do  servidor  serão coletadas  (como  uso  de  CPU  e  latência).

  4.  Experimentos  com  Carga  Constante:  Semelhantes  aos  experimentos  de  aumento  gradual  de  carga,  exceto  que agora  a  taxa  de  requisições  é  mantida  constante  em  um  nível  correspondente  a  cerca  de  70\%  de  utilização  de recursos  (seja  qual  for  o  fator  limitante  para  o  benchmark  em  questão).  Além  disso,  utilizamos  uma  única  série  de testes  em  vez  de  cinco,  pois  a  variância  foi  muito  menor  do  que  para  o  aumento  gradual  de  carga  em  nosso  ambiente  de  teste.

  5.  Validação  e  Experimentos  Adicionais:  Realize  experimentos  de  validação  e  adicionais  para  explicar  os  resultados  e  validar  o  trabalho.  Isso  inclui  a  criação  de  perfis  de  pilha,  a  análise  da  conexão  de  rede  e  a  avaliação  do  impacto do  software  de  criação  de  perfis.


  \section*{Página 30}

  Teste  de  E/S - atraso  fixo  de  100  ms.


  \section*{Página 31}

  Antes  de  cada  iteração,  a  máquina  atacante  enviará  uma  solicitação  HTTP  que  força  o  servidor  a  realizar  uma  coleta  de  lixo.


  \section*{Página 32}

  O  aquecimento  é  realizado  executando  um  teste  de  carga  prolongado  com  o  mesmo  método  de teste  antes  de  cada  iteração.  O  aquecimento  também  visa  forçar  a  compilação  JIT.  A  máquina  que  executa  o  servidor  possui  um  contador  de  limite  para  compilação  JIT  igual  a  10.000  (acessado  pelo  comando  \texttt{java -XX:+UnlockDiagnosticVMOptions -XX:+PrintFlagsFinal -version}),  o  que  significa  que  um  método  pode  ser  compilado  após  ser  chamado  esse  número  de  vezes.  

  Assim,  as  especificidades  do  aquecimento  (número  de  requisições)  serão  projetadas  para  chamar  os  métodos  mais  de  10.000  vezes  com  margem  de  segurança.  

 % Minha máquina: 10000

 % Máquina virtual: ???  
 % Executar \texttt{java -XX:+UnlockDiagnosticVMOptions -XX:+PrintFlagsFinal -version | grep CompileThreshold}

  Média  de  cinco  testes

  \section*{Página 33}

  Uma  máquina  hospedava  um  servidor  e  coletava  dados  de  perfil,  enquanto  a  outra  realizava  testes  de  carga  por  meio  do  Vegeta.  
  As  máquinas  estavam  conectadas  por  um  cabo  Ethernet  de  um  gigabit.

  A  máquina  cliente  coleta  as  seguintes métricas:  latência  (mínima,  máxima,  média,  percentis  50,  90,  95  e  99),  taxa  de  transferência,  taxa  de  amostragem,  respostas  do  servidor  (para  taxa  de  sucesso),  bytes  de  entrada  e  bytes  de  saída.  Isso  é  feito  através  do  Vegeta.

  O  servidor  coleta  métricas  referentes  à  utilização  da  CPU,  número  de  threads  (ativas  e  iniciadas)  e  heap  (tamanho  total  e  tamanho  utilizado).

  \section*{Página 34}

  \begin{itemize}
      \item Realizamos  cada  série  de  testes  cinco  vezes  e  calculamos  as  médias.  À  medida  que  cada  série  de  testes  era  repetida,  obtínhamos  uma  média  de  cinco  repetições.
      \item Um  teste  foi  interrompido  após  três  requisições  consecutivas  com  falha. 
      \item Atrasos  de  60  segundos  antes  da  coleta  de  lixo  e  20  segundos  depois.  Essa  configuração  foi  definida  empiricamente  por  meio  do  estudo  de  séries  temporais  do  VisualVM,  escolhendo  valores  que  permitissem  que  a  utilização  da  CPU  e  o  uso  de  memória  retornassem  aos  valores  nominais  entre  os  testes.
  \end{itemize}


  \section*{Página 35}

  Aquecimento:

  \begin{enumerate}
      \item 60  segundos  com  300  solicitações  por  segundo
      \item Coleta  manual  de  lixo
      \item Operação  em  modo  de  espera  por  20  segundos
      \item Repita  três  vezes  com  taxas  ajustadas  para  o  método  específico
  \end{enumerate}

  As  três  taxas  de  aquecimento  posteriores  são  ajustadas  para  70\%  da  taxa  máxima  do  método.

  Ataques:

  \begin{enumerate}
      \item Envie  solicitações  HTTP  com  a  taxa  'RATE'  durante  10  segundos.  Salve  os  resultados
      \item Operação  em  modo  de  repouso  por  60  segundos
      \item Coleta  manual  de  lixo
      \item Durma  por  20  segundos
      \item Aumente  a  'TAXA'  em  25/50  e  repita
  \end{enumerate}


  \section*{Página 36}

  Perfilamento  de  Pilha:

  Para  investigar  as  pilhas  de  chamadas foram utilizados FlameGraphs.

  O  procedimento  de  teste  seguiu  os  mesmos  protocolos  apresentados  na  seção  anterior,  incluindo extensos  aquecimentos.  O  teste  em  si,  no  entanto,  foi  mais  curto  (com  240  segundos).

  As  taxas  foram  ajustadas  manualmente  com  o  objetivo  de  não  sobrecarregar  os  servidores  durante  os  testes.


  \section*{Página 37}

  Conexão  de  Rede:

  Quando  os  testes  de  carga  foram  realizados  neste  estudo,  eles  foram  executados  até  que  o  servidor falhasse  ou  a  conexão  de  rede  fosse  interrompida.

  \section{Atenção}

  Scripts no appendix A, página 77

  Este script foi usado no cliente para enviar solicitações HTTP com Vegeta e salvar os resultados de forma organizada em um arquivo.

\end{document}
