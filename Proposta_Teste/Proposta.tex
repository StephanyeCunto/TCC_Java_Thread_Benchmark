\documentclass[12pt,a4paper]{article}

\usepackage[utf8]{inputenc}
\usepackage[brazil]{babel}
\usepackage{geometry}
\usepackage{setspace}
\usepackage{graphicx}
\usepackage{booktabs}
\usepackage{caption}
\usepackage{amsmath}
\usepackage{array}

\geometry{margin=2.5cm}
\onehalfspacing

\begin{document}

\section{Metodologia}

\subsection{Configuração do Experimento}
\begin{itemize}
    \item Linguagem: Java (JDK 21)
    \item Tipo de tarefas: I/O-bound
        \begin{itemize}
            \item Por que 30?
                \begin{itemize} 
                    \item O autor não diz
                \end{itemize}
            \item Por que 60? 
                \begin{itemize} 
                    \item O autor não realizou a etapa de warmup, a partir desse número, a margem de erro tornou-se estável.
                \end{itemize}

        \end{itemize}
    \item \textit{Warm-up}: a definir 
    \item Outliers: resultados extremos serão removidos da análise (utilizar boxplot)
    \item Retirar erros de rede
\end{itemize}

\subsection{Tarefas}
I/O-bound:
\begin{itemize}
    \item Servidor http
\end{itemize} 

\subsection{Métricas de Desempenho}

\begin{itemize}
    \item Tempo de execução
    \item Consumo de memória
    \item Uso de CPU
    \item Quatidade de threads/segundo
\end{itemize} 

Obs: Contar o número de requisições requisitadas e o número de requisições que chegaram ao servidor.
\end{document}
