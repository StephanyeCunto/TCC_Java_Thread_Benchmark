\documentclass[12pt,a4paper]{article}

\usepackage[utf8]{inputenc}
\usepackage[brazil]{babel}
\usepackage{geometry}
\usepackage{setspace}
\usepackage{graphicx}
\usepackage{booktabs}
\usepackage{caption}
\usepackage{amsmath}
\usepackage{array}

\geometry{margin=2.5cm}
\onehalfspacing

\begin{document}

\section{Metodologia}

\subsection{Configuração do Experimento}
\begin{itemize}
    \item Linguagem: Java (JDK 21 ou 25)
    \item Tipo de tarefas: CPU-bound e I/O-bound
    \item Execuções: 30 repetições por cenário
    \item \textit{Warm-up}: a definir (seguir abordagem similar à utilizada em \textit{“Avaliação dos mecanismos de concorrência na API do Java 8”})
    \item Outliers: resultados extremos serão removidos da análise
    \item Número de threads: ???
\end{itemize}

\subsection{Tarefas}
CPU-bound:
\begin{itemize}
    \item Cálculo de números primos em um intervalo.
    \item Operações matemáticas com matrizes (multiplicação, soma e transposição).
\end{itemize}
I/O-bound:
\begin{itemize}
    \item Simular com atraso com Thread.sleep
    \item Leitura de um arquivo
\end{itemize} 

\subsection{Métricas de Desempenho}

\begin{table}[h!]
    \centering
    \caption{Métricas utilizadas na avaliação de desempenho}
    \renewcommand{\arraystretch}{1.3}
    \begin{tabular}{|p{4cm}|p{10cm}|}
        \hline
        \textbf{Métrica} & \textbf{Descrição e Justificativa} \\ \hline
        Tempo de execução (\textmu s) &  Verifica se o modelo de execução influencia o tempo total de execução em tarefas puramente CPU-bound. \\ \hline
        Consumo de memória (MB) & Threads virtuais utilizam menos memória de pilha, permitindo maior escalabilidade. Essa métrica evidencia a eficiência no uso de recursos entre os dois modelos. \\ \hline
        Uso de CPU (\%) & Indica se há paralelismo efetivo ou contenção de recursos. Ajuda a avaliar o aproveitamento dos núcleos disponíveis e o impacto do agendamento de threads. \\ \hline
    \end{tabular}
\end{table}


\end{document}
