\DTLnewrow{artigos}
\DTLnewdbentry{artigos}{num}{1}
\DTLnewdbentry{artigos}{titulo}{Avaliação dos mecanismos de concorrência na API do Java 8}
\DTLnewdbentry{artigos}{producao} {Dissertação}
\DTLnewdbentry{artigos}{autor}{DANIEL JORGE DE OLIVEIRA AGUAS}
\DTLnewdbentry{artigos}{ano}{2015}
\DTLnewdbentry{artigos}{link}{https://core.ac.uk/download/pdf/302867260.pdf}
\DTLnewdbentry{artigos}{so}{Não citado}
%\DTLnewdbentry{artigos}{resumo}{Qui sint eu laboris minim ex nostrud esse in. Irure deserunt commodo aute ex non aliquip cillum elit velit enim cillum elit. Irure laboris et irure Lorem non cillum anim laboris tempor laboris sunt. Sint eiusmod commodo dolor dolor elit eiusmod minim magna. Dolor aliquip tempor laboris duis officia voluptate ad aliqua excepteur. Laboris culpa magna proident consequat.}
\DTLnewdbentry{artigos}{notas}{%
    \begin{itemize}
        \item API do Java 8 (Threads, Threadpools, ExecutorService, CountdownLach, ExecutorCompletionService e ForkJoinPools) 
        \item Utilizou os algoritmos de ordenação: Quicksort, Mergesort e Pidgeonholesort com 1000, 5000, 10000, 50000 e 10000 algarismos 
        \item Máquina virtual java utilizada: JVM Hotspot
        \item Itens que se procurou diferenciar nas máquinas testadas:
        \begin{itemize} 
            \item Quantidade de núcleos físicos 
            \item Dimensão da cache 
            \item Presença ou não de tecnologias como o HyperThreading e Turbo Boost nos processadores 
        \end{itemize}
        \item Fatores ignorados devido à dificuldade de obtenção de grandes quantidades de máquinas:
        \begin{itemize} 
            \item Diferentes Sistemas Operativos e diferentes versões deles 
        \end{itemize}
        \item Máquinas utilizados :
        \begin{itemize}
            \item Intel® Core™ T4300 – 2,1 GHz, 2 núcleos, 1 MB, Nenhuma, 4gb de ram
            \item Intel® Core™ i5-4260U – 1,4 GHz, 2 núcleos, 3 MB, Hyper-Threading, Turbo Boost (2.8GHz), 4gb de ram
            \item Intel® Core™ i7-3610QM – 2,3 GHz, 4 núcleos, 6 MB, Hyper-Threading, Turbo Boost (2.8GHz), 6gb de ram
        \end{itemize}
        \item Devido ao consumo de memória, foi necessário aumentar a heap da JVM para 4000 MB. A heap é a área de memória onde os objetos criados em tempo de execução são armazenados.O garbage collector monitora esses objetos e libera memória removendo aqueles que não são mais utilizados, evitando que a aplicação fique sem memória. 
        \item Para produzir o ambiente desejado foram efetuadas algumas alterações na ferramenta de análise JProfiler.
        \item Rentável, se a duração base de execução do algoritmo permitir que o overhead de tempo com a criação de paralelismo seja compensado, e que o impacto no consumo de memória depende da estrutura do algoritmo.
        \item Trabalhos futuros:
        \begin{itemize}
            \item Executar testes em máquinas com processadores que possuam maior capacidade de threads, incluindo processadores com quatro núcleos físicos, para comparar resultados em ambientes com e sem Hyper-Threading.  
            \item Realizar o estudo em um sistema operacional que facilite o uso de \textit{affinity}, permitindo maior dedicação do CPU aos testes.  
            \item Implementar e avaliar outras ferramentas da API do Java 8, ampliando as conclusões sobre programação concorrente.  
        \end{itemize}
    \end{itemize}
}

\DTLnewrow{artigos}
\DTLnewdbentry{artigos}{num}{2}
\DTLnewdbentry{artigos}{titulo}{Uma análise comparativa entre Threads e Green Threads no Java}
\DTLnewdbentry{artigos}{producao}{TCC/Artigo}
\DTLnewdbentry{artigos}{autor}{Hiarly Fernandes de Souto, Thiago Emmanuel Pereira}
\DTLnewdbentry{artigos}{ano}{2024}
\DTLnewdbentry{artigos}{link}{https://dspace.sti.ufcg.edu.br/bitstream/riufcg/38147/1/HIARLY\%20FERNANDES\%20DE\%20SOUTO-ARTIGO-CIÊNCIA\%20DA\%20COMPUTAÇÃO-CEEI\%20(2024).pdf}
\DTLnewdbentry{artigos}{so}{Ubuntu}
%\DTLnewdbentry{artigos}{resumo}{Qui sint eu laboris minim ex nostrud esse in. Irure deserunt commodo aute ex non aliquip cillum elit velit enim cillum elit. Irure laboris et irure Lorem non cillum anim laboris tempor laboris sunt. Sint eiusmod commodo dolor dolor elit eiusmod minim magna. Dolor aliquip tempor laboris duis officia voluptate ad aliqua excepteur. Laboris culpa magna proident consequat.}
\DTLnewdbentry{artigos}{notas}{
    \begin{itemize}
        \item Green Threads, chamada de Virtual Threads, design de threads no qual o escalonamento é feito pela própria aplicação por meio de uma biblioteca ou framework. 
        \item Testes de Criação, Inicialização, Junção e Troca de Contexto.
        \item Threads nativas são eficazes em termos de paralelismo real, elas também são caras em termos de recursos e podem ser difíceis de usar em algumas situações de programação.
        \item Virtual Threads são mais leves que as threads do kernel em termos de uso de memória, e a sobrecarga de troca de contexto entre elas é muito baixa. Milhões de Virtual Threads podem ser criadas em uma única instância de JVM.
        \item Máquina utilizado: AMD® Ryzen 7 3700u, 20GB de memória RAM, disco com capacidade de 256GB
        \item Criou-se 100.000 threads e mediu–se o tempo de execução para a criação desses objetos
        \item Iniciou-se um total de 10.000 threads que não realizavam operações específicas
        \item Cada thread aguarda por 100 milissegundos, invoca o método Thread.yield() (Mudar Contexto)
        \item Instanciar Threads, Iniciar Threads,Join Thread, Mudar Contexto a Thread Virtual é mais rápido
        \item Ausência de diversidade nas configurações de infraestrutura
        \item A realização dos testes do estudo em ambientes distintos poderia levar a resultados divergentes, uma vez que a JVM (Java Virtual Machine) poderia realizar otimizações específicas de acordo com o hardware da máquina utilizada
    \end{itemize}%
}

\DTLnewrow{artigos}
\DTLnewdbentry{artigos}{num}{3}
\DTLnewdbentry{artigos}{titulo}{Comparison of Concurrency Technologies in Java}
\DTLnewdbentry{artigos}{producao}{Dissertação}
\DTLnewdbentry{artigos}{autor}{Elias Gustafsson, Oliver Nederlund Persson}
\DTLnewdbentry{artigos}{ano}{2024}
\DTLnewdbentry{artigos}{link}{https://lup.lub.lu.se/luur/download?func=downloadFile\&recordOId=9166685\&fileOId=9166687}
\DTLnewdbentry{artigos}{so}{MacOS}
%\DTLnewdbentry{artigos}{resumo}{Qui sint eu laboris minim ex nostrud esse in. Irure deserunt commodo aute ex non aliquip cillum elit velit enim cillum elit. Irure laboris et irure Lorem non cillum anim laboris tempor laboris sunt. Sint eiusmod commodo dolor dolor elit eiusmod minim magna. Dolor aliquip tempor laboris duis officia voluptate ad aliqua excepteur. Laboris culpa magna proident consequat.}
\DTLnewdbentry{artigos}{notas}{
    \begin{itemize}
        \item Foram realizadas análises de pilha, tráfego de rede e impacto de ferramentas de profiling.
        \item Reactive Systems são sistemas assíncronos, escaláveis, resilientes e responsivos, que usam fluxos de dados e mensagens para lidar com alta carga e eventos em tempo real, garantindo desempenho mesmo diante de falhas.
        \item Pergunta 1: Quais são as diferenças entre essas diferentes técnicas de simultaneidade dentro da JVM em sistemas de alta carga?
        \item Pergunta 2: Threads virtuais são uma alternativa viável para substituir threads de plataforma e fluxos de dados assíncronos em aplicações de alta carga?
        \item Maquina utlizada: (MacBook Pro 2019) 2.6 GHz-6 core Intel Core i7, Memory 16 GB 2667 MHz DDR4, MacOS 14.2.1.
        \item Desafios em Java:
        \begin{itemize}
            \item Garbage Collector (GC) — coleta de lixo pode causar pausas variáveis. 
            \item JRE e JVM — diferentes máquinas virtuais podem apresentar comportamentos distintos.
            \item Just-In-Time (JIT) Compilation — otimizações dinâmicas podem alterar resultados entre execuções.
        \end{itemize}
        \item Três servidores foram implementados utilizando threads da plataforma, threads virtuais e programação reativa.
        \item Testes com taxa de requisições constante foram conduzidos para avaliar a estabilidade.
    \end{itemize}%
}

\DTLnewrow{artigos}
\DTLnewdbentry{artigos}{num}{4}
\DTLnewdbentry{artigos}{titulo}{Analise Comparativa Entre Java e Kotlin No Contexto Da Programação Concorrente}
\DTLnewdbentry{artigos}{producao}{TCC/Artigo}
\DTLnewdbentry{artigos}{autor}{}
\DTLnewdbentry{artigos}{ano}{2024}
\DTLnewdbentry{artigos}{link}{https://bib.pucminas.br/pergamumweb/download/611E6580-CF2F-40D3-87CC-174C0B58E999.pdf}
\DTLnewdbentry{artigos}{so}{}
%\DTLnewdbentry{artigos}{resumo}{Qui sint eu laboris minim ex nostrud esse in. Irure deserunt commodo aute ex non aliquip cillum elit velit enim cillum elit. Irure laboris et irure Lorem non cillum anim laboris tempor laboris sunt. Sint eiusmod commodo dolor dolor elit eiusmod minim magna. Dolor aliquip tempor laboris duis officia voluptate ad aliqua excepteur. Laboris culpa magna proident consequat.}
\DTLnewdbentry{artigos}{notas}{
    \begin{itemize}
        \item 
        \item
    \end{itemize}%
}

\DTLnewrow{artigos}
\DTLnewdbentry{artigos}{num}{5}
\DTLnewdbentry{artigos}{titulo}{Programação Concorrente em Rust e Java: Uma análise comparativa de segurança, produção e performance}
\DTLnewdbentry{artigos}{producao}{}
\DTLnewdbentry{artigos}{autor}{}
\DTLnewdbentry{artigos}{ano}{}
\DTLnewdbentry{artigos}{link}{https://www.researchgate.net/profile/Andre-Marcos-Silva/publication/378609476_Programacao_Concorrente_em_Rust_e_Java_Uma_analise_comparativa_de_seguranca_producao_e_performance/links/67297b49db208342deeaa1c7/Programacao-Concorrente-em-Rust-e-Java-Uma-analise-comparativa-de-seguranca-producao-e-performance.pdf}
\DTLnewdbentry{artigos}{so}{}
%\DTLnewdbentry{artigos}{resumo}{Qui sint eu laboris minim ex nostrud esse in. Irure deserunt commodo aute ex non aliquip cillum elit velit enim cillum elit. Irure laboris et irure Lorem non cillum anim laboris tempor laboris sunt. Sint eiusmod commodo dolor dolor elit eiusmod minim magna. Dolor aliquip tempor laboris duis officia voluptate ad aliqua excepteur. Laboris culpa magna proident consequat.}
\DTLnewdbentry{artigos}{notas}{
    \begin{itemize}
        \item 
        \item
    \end{itemize}%
}

\DTLnewrow{artigos}
\DTLnewdbentry{artigos}{num}{6}
\DTLnewdbentry{artigos}{titulo}{}
\DTLnewdbentry{artigos}{producao}{}
\DTLnewdbentry{artigos}{autor}{}
\DTLnewdbentry{artigos}{ano}{}
\DTLnewdbentry{artigos}{link}{}
\DTLnewdbentry{artigos}{so}{}
%\DTLnewdbentry{artigos}{resumo}{Qui sint eu laboris minim ex nostrud esse in. Irure deserunt commodo aute ex non aliquip cillum elit velit enim cillum elit. Irure laboris et irure Lorem non cillum anim laboris tempor laboris sunt. Sint eiusmod commodo dolor dolor elit eiusmod minim magna. Dolor aliquip tempor laboris duis officia voluptate ad aliqua excepteur. Laboris culpa magna proident consequat.}
\DTLnewdbentry{artigos}{notas}{
    \begin{itemize}
        \item 
        \item
    \end{itemize}%
}

\iffalse
\DTLnewrow{artigos}
\DTLnewdbentry{artigos}{num}{5}
\DTLnewdbentry{artigos}{titulo}{}
\DTLnewdbentry{artigos}{producao}{}
\DTLnewdbentry{artigos}{autor}{}
\DTLnewdbentry{artigos}{ano}{}
\DTLnewdbentry{artigos}{link}{}
\DTLnewdbentry{artigos}{so}{}
%\DTLnewdbentry{artigos}{resumo}{Qui sint eu laboris minim ex nostrud esse in. Irure deserunt commodo aute ex non aliquip cillum elit velit enim cillum elit. Irure laboris et irure Lorem non cillum anim laboris tempor laboris sunt. Sint eiusmod commodo dolor dolor elit eiusmod minim magna. Dolor aliquip tempor laboris duis officia voluptate ad aliqua excepteur. Laboris culpa magna proident consequat.}
\DTLnewdbentry{artigos}{notas}{
    \begin{itemize}
        \item 
        \item
    \end{itemize}%
}
\fi