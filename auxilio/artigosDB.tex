\DTLnewrow{artigos}
\DTLnewdbentry{artigos}{num}{1}
\DTLnewdbentry{artigos}{titulo}{Avaliação dos mecanismos de concorrência na API do Java 8}
\DTLnewdbentry{artigos}{producao} {Dissertação}
\DTLnewdbentry{artigos}{autor}{DANIEL JORGE DE OLIVEIRA AGUAS}
\DTLnewdbentry{artigos}{ano}{2015}
\DTLnewdbentry{artigos}{link}{https://core.ac.uk/download/pdf/302867260.pdf}
\DTLnewdbentry{artigos}{so}{Não citado}
%\DTLnewdbentry{artigos}{resumo}{Qui sint eu laboris minim ex nostrud esse in. Irure deserunt commodo aute ex non aliquip cillum elit velit enim cillum elit. Irure laboris et irure Lorem non cillum anim laboris tempor laboris sunt. Sint eiusmod commodo dolor dolor elit eiusmod minim magna. Dolor aliquip tempor laboris duis officia voluptate ad aliqua excepteur. Laboris culpa magna proident consequat.}
\DTLnewdbentry{artigos}{notas}{%
    \begin{itemize}
        \item API do Java 8 (Threads, Threadpools, ExecutorService, CountdownLach, ExecutorCompletionService e ForkJoinPools) 
        \item Utilizou os algoritmos de ordenação: Quicksort, Mergesort e Pidgeonholesort com 1000, 5000, 10000, 50000 e 10000 algarismos 
        \item Máquina virtual java utilizada: JVM Hotspot
        \item Itens que se procurou diferenciar nas máquinas testadas:
        \begin{itemize} 
            \item Quantidade de núcleos físicos 
            \item Dimensão da cache 
            \item Presença ou não de tecnologias como o HyperThreading e Turbo Boost nos processadores 
        \end{itemize}
        \item Fatores ignorados devido à dificuldade de obtenção de grandes quantidades de máquinas:
        \begin{itemize} 
            \item Diferentes Sistemas Operativos e diferentes versões deles 
        \end{itemize}
        \item Processadores utilizados :
        \begin{itemize}
            \item Intel® Core™ T4300 – 2,1 GHz, 2 núcleos, 1 MB, Nenhuma, 4gb de ram
            \item Intel® Core™ i5-4260U – 1,4 GHz, 2 núcleos, 3 MB, Hyper-Threading, Turbo Boost (2.8GHz), 4gb de ram
            \item Intel® Core™ i7-3610QM – 2,3 GHz, 4 núcleos, 6 MB, Hyper-Threading, Turbo Boost (2.8GHz), 6gb de ram
        \end{itemize}
        \item Devido ao consumo de memória, foi necessário aumentar a heap da JVM para 4000 MB. A heap é a área de memória onde os objetos criados em tempo de execução são armazenados.O garbage collector monitora esses objetos e libera memória removendo aqueles que não são mais utilizados, evitando que a aplicação fique sem memória. 
        \item Para produzir o ambiente desejado foram efetuadas algumas alterações na ferramenta de análise JProfiler.
        \item Rentável, se a duração base de execução do algoritmo permitir que o overhead de tempo com a criação de paralelismo seja compensado, e que o impacto no consumo de memória depende da estrutura do algoritmo.
        \item Trabalhos futuros:
        \begin{itemize}
            \item Executar testes em máquinas com processadores que possuam maior capacidade de threads, incluindo processadores com quatro núcleos físicos, para comparar resultados em ambientes com e sem Hyper-Threading.  
            \item Realizar o estudo em um sistema operacional que facilite o uso de \textit{affinity}, permitindo maior dedicação do CPU aos testes.  
            \item Implementar e avaliar outras ferramentas da API do Java 8, ampliando as conclusões sobre programação concorrente.  
        \end{itemize}
    \end{itemize}
}

\DTLnewrow{artigos}
\DTLnewdbentry{artigos}{num}{2}
\DTLnewdbentry{artigos}{titulo}{Comparison of Concurrency Technologies in Java}
\DTLnewdbentry{artigos}{producao}{Dissertação}
\DTLnewdbentry{artigos}{autor}{Elias Gustafsson, Oliver Nederlund Persson}
\DTLnewdbentry{artigos}{ano}{2024}
\DTLnewdbentry{artigos}{link}{https://lup.lub.lu.se/luur/download?func=downloadFile\&recordOId=9166685\&fileOId=9166687}
\DTLnewdbentry{artigos}{so}{MacOS}
%\DTLnewdbentry{artigos}{resumo}{Qui sint eu laboris minim ex nostrud esse in. Irure deserunt commodo aute ex non aliquip cillum elit velit enim cillum elit. Irure laboris et irure Lorem non cillum anim laboris tempor laboris sunt. Sint eiusmod commodo dolor dolor elit eiusmod minim magna. Dolor aliquip tempor laboris duis officia voluptate ad aliqua excepteur. Laboris culpa magna proident consequat.}
\DTLnewdbentry{artigos}{notas}{
    \begin{itemize}
        \item Foram realizadas análises de pilha, tráfego de rede e impacto de ferramentas de profiling.
        \item Reactive Systems são sistemas assíncronos, escaláveis, resilientes e responsivos, que usam fluxos de dados e mensagens para lidar com alta carga e eventos em tempo real, garantindo desempenho mesmo diante de falhas.
        \item Pergunta 1: Quais são as diferenças entre essas diferentes técnicas de simultaneidade dentro da JVM em sistemas de alta carga?
        \item Pergunta 2: Threads virtuais são uma alternativa viável para substituir threads de plataforma e fluxos de dados assíncronos em aplicações de alta carga?
        \item Maquina utlizada: (MacBook Pro 2019) 2.6 GHz-6 core Intel Core i7, Memory 16 GB 2667 MHz DDR4, MacOS 14.2.1.
        \item Desafios em Java:
        \begin{itemize}
            \item Garbage Collector (GC) — coleta de lixo pode causar pausas variáveis. 
            \item JRE e JVM — diferentes máquinas virtuais podem apresentar comportamentos distintos.
            \item Just-In-Time (JIT) Compilation — otimizações dinâmicas podem alterar resultados entre execuções.
        \end{itemize}
        \item Três servidores foram implementados utilizando threads da plataforma, threads virtuais e programação reativa.
        \item Testes com taxa de requisições constante foram conduzidos para avaliar a estabilidade.
    \end{itemize}%
}