% Artigo 1
\DTLnewrow{artigos}
\DTLnewdbentry{artigos}{num}{1}
\DTLnewdbentry{artigos}{titulo}{Programação Paralela e Distribuída em Java}
\DTLnewdbentry{artigos}{autor}{Lucilene Baeta Ferrão}
\DTLnewdbentry{artigos}{ano}{2005}
\DTLnewdbentry{artigos}{link}{https://ri.unipac.br/repositorio/wp-content/uploads/tainacan-items/282/193827/Lucilene-Baeta-Ferrao-Programacao-Paralela-e-Distribuida-em-Java-COMPUTACAO-2005.pdf}
\DTLnewdbentry{artigos}{so}{ }
\DTLnewdbentry{artigos}{resumo}{O artigo apresenta conceitos de programação paralela e distribuída utilizando Java,abordando threads, sincronização e comunicação entre processos.}
\DTLnewdbentry{artigos}{notas}{Diferença entre paralelismo e distribuído,Limitações do Java em 2005,Importância da sincronização}

% Artigo 2
\DTLnewrow{artigos}
\DTLnewdbentry{artigos}{num}{2}
\DTLnewdbentry{artigos}{titulo}{[Título do Artigo]}
\DTLnewdbentry{artigos}{autor}{[Autor]}
\DTLnewdbentry{artigos}{ano}{[Ano]}
\DTLnewdbentry{artigos}{link}{[URL]}
\DTLnewdbentry{artigos}{so}{[SO usado]}
\DTLnewdbentry{artigos}{resumo}{Resumo do artigo 2 a ser preenchido.}
\DTLnewdbentry{artigos}{notas}{Primeira anotação,Segunda anotação}

% Artigo 3
\DTLnewrow{artigos}
\DTLnewdbentry{artigos}{num}{3}
\DTLnewdbentry{artigos}{autor}{[Autor]}
\DTLnewdbentry{artigos}{titulo}{[Título do Artigo]}
\DTLnewdbentry{artigos}{ano}{[Ano]}
\DTLnewdbentry{artigos}{link}{[URL]}
\DTLnewdbentry{artigos}{so}{[SO usado]}
\DTLnewdbentry{artigos}{resumo}{Resumo do artigo 3 a ser preenchido.}
\DTLnewdbentry{artigos}{notas}{Apenas uma nota}