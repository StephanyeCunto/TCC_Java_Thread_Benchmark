% Adaptado para uso no DACC - IF Sudeste MG, campus Rio Pomba
% ABNT NBR 6023/2025, 14724/2024, 10520/2023 - Trabalhos academicos
%Atualizado em agosto de 2025 (Alessandra Martins Coelho)

\documentclass[12pt, openright, oneside, a4paper, english,	french,	spanish, brazil	]{abntex2}

\usepackage{fontspec}
\setmainfont{Arial}
\setsansfont{Arial}
\bibliographystyle{abntex2-alf-antigo}
\usepackage[alf]{abntex2cite}	
\renewcommand\UrlFont{\normalfont} 
\usepackage{lmodern}
\usepackage[T1]{fontenc}		
\usepackage[utf8]{inputenc}		
\usepackage{indentfirst}		
\usepackage{color}				
\usepackage{graphicx}			
\usepackage{microtype} 			
\usepackage{lipsum}				
\usepackage{verbatimbox}
\usepackage{cmap}				
\usepackage{lastpage}			
\usepackage{xcolor}	
\usepackage{underscore}         
\usepackage{multicol}          
\usepackage{placeins}
\usepackage{amsmath}
\usepackage{amssymb}
\usepackage{hyperref}
\usepackage{setspace}
\usepackage{pdfpages}
\usepackage{float}
\usepackage{caption}

\titulo{ESCREVER O TÍTULO AQUI} 
\autor{STEPHANYE CRISTINE ANTUNES DE CUNTO} 
\local{RIO POMBA - MG}
\orientador{ME. BIANCA PORTES DE CASTRO} 
\coorientador{DR. JOSÉ RUI CASTRO DE SOUSA}
\tipotrabalho{Trabalho de Conclusão de Curso}
\preambulo{Trabalho de Conclusão de curso apresentado ao \textit{Campus} Rio Pomba, do Instituto Federal de Educação, Ciência e Tecnologia do Sudeste de Minas Gerais, como parte das exigências do curso de Bacharelado em Ciência da Computação para a obtenção do título de Bacharel em Ciência da Computação.}

\definecolor{blue}{RGB}{41,5,195}

\makeatletter
    \hypersetup{ 
        pdftitle={\@title}, pdfauthor={\@author}, pdfsubject={\imprimirpreambulo}, pdfcreator={\@author},
        pdfkeywords={ciência da computação}{trabalho acadêmico}{constraint programming}{programação por restrições}{examination timetabling}, 
        colorlinks=true, linkcolor=black, citecolor=black, filecolor=black, urlcolor=black, bookmarksdepth=4       	
    }
\makeatother

\makeatletter
    \setlength{\@fptop}{5pt}
\makeatother

\newcommand{\ignore}[1]{}

\newcommand{\quadroname}{Quadro}
\newcommand{\listofquadrosname}{Lista de quadros}

%\newfloat[chapter]{quadro}{loq}{\quadroname} % ver depois erro abntex2.cls
\newfloat{quadro}{htbp}{loq}[section]
\newlistof{listofquadros}{loq}{\listofquadrosname}
\newlistentry{quadro}{loq}{0}

\setfloatadjustment{quadro}{\centering}
\counterwithout{quadro}{chapter}
\renewcommand{\cftquadroname}{\quadroname\space} 
\renewcommand*{\cftquadroaftersnum}{\hfill--\hfill}

\setfloatlocations{quadro}{hbtp} 

\setlength{\parindent}{1.3cm}
\setlength{\parskip}{0.2cm}  

\makeindex

\DeclareCaptionLabelSeparator{barr}{\space\textendash\space}

% quadro sempre com legenda acima
%\captionsetup[quadro]{position=top}% 

% no preâmbulo, junto ao seu setup de caption
\captionsetup[quadro]{%
  position=top, justification=raggedright, singlelinecheck=false         
}

\begin{document}

    \captionsetup{ font= small, labelfont=bf, textfont=small, labelsep=barr, justification=justified, singlelinecheck=false }

    \selectlanguage{brazil}

    \frenchspacing 

    \begin{center}
        \textbf{ 
            INSTITUTO FEDERAL DE EDUCAÇÃO, CIÊNCIA E TECNOLOGIA DO SUDESTE DE MINAS GERAIS - CAMPUS RIO POMBA
        }
    \end{center}

    \imprimircapa
    \imprimirfolhaderosto*

    \begin{fichacatalografica}
        \vspace*{\fill}					
        \hrule							
        \begin{center}					
            \begin{minipage}[c]{12.5cm}		
            
                FICHA CATALOGRÁFICA TEMPORÁRIA \\
                \imprimirautor
                
                \hspace{0.5cm} \imprimirtitulo  / \imprimirautor. --
                \imprimirlocal, \imprimirdata-

                \hspace{0.5cm} \imprimirorientadorRotulo~\imprimirorientador\\
                
                \hspace{0.5cm}
                \parbox[t]{\textwidth}{%
                    \normalsize
                    \imprimirtipotrabalho~--~Instituto Federal de Educação, Ciência e Tecnologia do Sudeste de Minas, Campus Rio Pomba
                }

            \end{minipage}
        \end{center}
        \hrule
    \end{fichacatalografica}

    \begin{folhadeaprovacao}

        \begin{center}
            {\ABNTEXchapterfont\normalsize\imprimirautor}

            \vspace*{\fill}\vspace*{\fill}
            {\ABNTEXchapterfont\bfseries\normalsize\imprimirtitulo}
            \vspace*{\fill}
            
            \hspace{.45\textwidth}
            \begin{minipage}{.5\textwidth}
                \imprimirpreambulo
            \end{minipage}%
            \vspace*{\fill}
        \end{center}
            
        Trabalho aprovado em XX de XXXXX de XXXX.

        \assinatura{\textbf{\imprimirorientador} \\Orientadora, IF Sudeste MG - Rio Pomba} 
        \assinatura{\textbf{\imprimircoorientador} \\Coorientador, IF Sudeste MG - Rio Pomba}
        \assinatura{\textbf{TÍTULO E NOME DO MEMBRO DA BANCA} \\ IF Sudeste MG - Rio Pomba}
        
        \begin{center}
            \vspace*{0.5cm}
            {\normalsize\imprimirlocal}
            \par
            {\normalsize\imprimirdata}
            \vspace*{1cm}
        \end{center}
    
    \end{folhadeaprovacao}

    \begin{dedicatoria}
        \vspace*{\fill}
        \begin{flushright}
            \hspace*{4cm}
            \parbox{11.2cm}{
                \raggedleft 
                \textit{
                }
            }
        \end{flushright}
    \end{dedicatoria}

    \begin{agradecimentos}
        Agradecimentos
    \end{agradecimentos}

    \begin{epigrafe}
        \vspace*{\fill}
     	\begin{flushright}
      		\textit{
                ``Os olhos não são apenas\\
      		    o espelho da alma,\\
      		    mas também do corpo.''\\
      		    (Ignatz von Peczelly, 1989)
            }
     	\end{flushright}
     \end{epigrafe}

    \begin{resumo}
        O resumo é um texto breve que apresenta, de forma clara e objetiva, os principais elementos da monografia. Ele deve permitir que o leitor compreenda rapidamente sobre o que é o trabalho, qual foi a abordagem adotada e quais foram os resultados e conclusões.
        Na ABNT (NBR 6028), recomenda-se que o resumo seja escrito em parágrafo único, sem subdivisões e sem citações diretas, geralmente com 150 a 500 palavras. O que incluir no resumo: Tema e objetivo – Apresente o assunto principal e o objetivo geral do trabalho; 
            Metodologia – Informe de forma resumida o método, técnicas ou procedimentos utilizados; Resultados – Destaque os resultados mais relevantes obtidos na pesquisa; 
            Conclusão – Apresente a principal contribuição ou conclusão do estudo. Dicas importantes: Escreva no tempo passado, já que o trabalho foi realizado. Evite usar abreviações pouco conhecidas ou siglas sem explicação. Não insira informações que não estejam no corpo do trabalho. Revise para garantir clareza, coerência e objetividade.
        Lembre-se de incluir palavras-chave logo abaixo do resumo (entre três e cinco termos que representem bem o conteúdo do trabalho, separados com ponto e vírgula). Importante: O resumo deve ser escrito após a conclusão do trabalho, quando todos os resultados já estão definidos. Assim, será fiel ao conteúdo final da monografia.

        \noindent \textbf{Palavras-Chave:} 
    \end{resumo}


    \begin{resumo}[Abstract]
        \begin{otherlanguage*}{english}
            Tradução do resumo para a língua inglesa.
                
            \noindent\textbf{Key-words}: 
        \end{otherlanguage*}
    \end{resumo}

    \pdfbookmark[0]{\listfigurename}{lof}
    \listoffigures*
    \cleardoublepage

    \pdfbookmark[0]{\listofquadrosname}{loq}
    \listofquadros*
    \cleardoublepage
    \pdfbookmark[0]{\listtablename}{lot}
    \listoftables*
    \cleardoublepage

    \begin{siglas}
        \item[CPU]     Unidade Central de Processamento
        \item[I/O]     Entrada/Saída
        \item[JVM]     Java Virtual Machine
        \item[SO]      Sistema Operacional
        \item[SPD]     Sistemas Paralelos E Distribuídos


        \item[XXX]     INSERIR EM ORDEM ALFABÉTICA

    \end{siglas}

    \begin{simbolos}
        \item[$ \Lambda $] Lambda
    \end{simbolos}

    \pdfbookmark[0]{\contentsname}{toc}
    \tableofcontents*
    \cleardoublepage

    \textual
    \setcounter{page}{14} 
 
    \chapter{Introdução}
    \markright{Introdução}
    \label{chapter:introducao}
    
        Uma thread é a menor unidade de processamento existente dentro de um processo. Cada processo é capaz de conter múltiplas threads, permitindo a execução simultânea de diferentes partes de um programa ou de diferentes tarefas \cite{silberschatz2018operating}.

        A utilização de threads oferece diversos benefícios em problemas que envolvem tarefas de alto processamento (\textit{CPU-bound}) e operações de entrada e saída (\textit{I/O-bound}). Além disso, em sistemas reativos e servidores,threads permitem que aplicações atendam múltiplos usuários simultaneamente,mantendo respostas rápidas e contínuas mesmo sob alta carga, garantindo desempenho eficiente e melhor experiência ao usuário. Assim, o uso de threads é aplicado como estratégia para otimizar recursos computacionais e melhorar a eficiência de sistemas em diferentes contextos.
        
        O relatório \textit{State of the Octoverse 2024} \cite{github2024octoverse} demonstra que a linguagem de programação Java está entre as cinco linguagens mais utilizadas na plataforma, o que reforça sua importância como uma das principais tecnologias do desenvolvimento de software.

        No lançamento da versão 19 do Java, foram introduzidas as threads virtuais, que, diferente das threads tradicionais, são gerenciadas pela \textit{Java Virtual Machine} (JVM). Enquanto o escalonamento das threads tradicionais é realizado pelo sistema operacional (SO), determinando quando cada thread é executada, o escalonamento das threads virtuais é feito pela própria JVM, podendo apresentar comportamentos distintos \cite{oracle_virtual_threads}. 

        O uso de threads tradicionais pode gerar \textit{overhead} e limitar a escalabilidade. Dessa forma, as threads virtuais surgem como alternativa para contornar essas limitações. Este trabalho investiga como a utilização de threads virtuais impacta o desempenho e a escalabilidade de aplicações concorrentes.

        O objetivo deste trabalho é analisar as diferenças de desempenho entre threads tradicionais e threads virtuais, avaliando como cada abordagem impacta a execução de aplicações concorrentes.

        Para alcançar o objetivo geral, foram definidos os seguintes objetivos específicos:

        \begin{itemize}
            \item Compreender os fundamentos teóricos da programação concorrente e das diferentes abordagens de threads;
            \item Implementar protótipos de aplicações concorrentes utilizando ambas as abordagens;
            \item Medir e comparar métricas de desempenho;
            \item Analisar os resultados obtidos, identificando vantagens e limitações.
        \end{itemize}

        A escolha deste tema parte do interesse gerado durante a disciplina de Sistemas Paralelos e Distribuídos (SPD), na qual foram estudados os conceitos fundamentais de concorrência, paralelismo e gerenciamento de threads. Além disso, o apreço pessoal pela linguagem Java contribuiu para a definição do tema, considerando que a plataforma tem investido em melhorias relacionadas à programação concorrente. Dessa forma, esse trabalho se justifica pelo interesse acadêmico adquirido ao longo da disciplina e pela pertinência de investigar uma inovação recente da linguagem Java, contribuindo para a compreensão de como esse novo modelo de threads pode aprimorar o desempenho de aplicações concorrentes.  
        
        Este trabalho está organizado da seguinte forma:

        \begin{itemize}
            \item \hyperref[chapter:introducao]{\textbf{Seção 1 - Introdução:} Apresenta o tema, os objetivos e a justificativa.}
            
            \item \hyperref[chapter:fundamentacaoTeorica]{\textbf{Seção 2 - Fundamentação teórica:} Expõe os conceitos de concorrência e paralelismo e descreve o funcionamento de processos, threads tradicionais e threads virtuais.}
            
            \item \hyperref[chapter:trabalhos]{\textbf{Seção 3 - Trabalhos Relacionados:} Revisa estudos anteriores sobre o desempenho de threads tradicionais e threads virtuais.}
            
            \item \hyperref[chapter:metodologia]{\textbf{Seção 4 - Metodologia:} Detalha como os testes foram conduzidos e como os dados foram coletados.}
            
            \item \hyperref[chapter:resultados]{\textbf{Seção 5 - Resultados e Discussão:} Apresenta e analisa os dados obtidos durante os testes.}
           
            \item \hyperref[chapter:conclusao]{\textbf{Seção 6 - Conclusão:} Expõe as considerações finais do estudo.}
        \end{itemize}


        Por fim, são listadas as referências bibliográficas utilizadas no trabalho, seguidas pelos apêndices e anexos que complementam o estudo.


    %A introdução é a primeira parte do texto e tem como objetivo apresentar, de forma clara e objetiva, o tema e o contexto da pesquisa. Ela deve despertar o interesse do leitor e situá-lo quanto à relevância do trabalho. Geralmente, a introdução contém:

    %    Contextualização – Apresente o tema no cenário em que ele está inserido, destacando dados, fatos ou informações que ajudem a compreender sua importância.

    %   Problema de Pesquisa – Indique claramente a questão que motivou o estudo, explicitando qual lacuna ou necessidade será abordada.

    %   Justificativa – Explique a relevância do trabalho, tanto acadêmica quanto prática, e por que ele merece ser desenvolvido.

    %   Objetivos – Informe o objetivo geral e, se necessário, os objetivos específicos, que detalham as etapas para alcançar o resultado pretendido.

    %   Estrutura do Trabalho – Descreva brevemente como o texto está organizado, citando os capítulos ou seções.

    %   A introdução não deve apresentar resultados, conclusões ou discussões. Ela prepara o terreno para o desenvolvimento do trabalho.
    %   Normalmente, é escrita no final da monografia, quando o autor já tem clareza total sobre todo o conteúdo.

    % ---------------------------------------------------------------------------------------------
    % Fundamentação Teórica
    % ---------------------------------------------------------------------------------------------

    \setcounter{chapter}{1}
    \chapter{Fundamentação Teórica}
    \label{chapter:fundamentacaoTeorica}

        Exemplo de citação no final do texto \cite{pecb2022iso27002}. 

        Exemplo de citação dentro do texto \citeonline{pecb2022iso27002}.

        Exemplo de uma Figura. Use ref para chamá-la no texto. Figura \ref{fig:figura1}.
        
    \begin{figure}[h]
        \centering
        \begin{minipage}{0.5\textwidth}
            \phantomsection
            \fontsize{10}{12}\selectfont
            \caption{Pilares da Segurança da Informação.}
            \includegraphics[width=1\linewidth]{imagens/Figura 1.png}
            \caption*{\raggedright\small \textbf{Fonte}: \citeonline{bughunt2023triade}.}
            \label{fig:figura1}
        \end{minipage}
    \end{figure}

    \section{Para criar um título} 
        Sempre inserir um texto entre os Títulos

    \subsection{mais um título}

    Um exemplo de Quadro (Quadro \ref{quad: quadro1}). 

    \begin{quadro}[ht]
    \centering
    \begin{minipage}{0.80\textwidth}
    \phantomsection
    \fontsize{10}{12}\selectfont
    \caption{Panorama dos ataques baseados em engenharia social.} 
    \begin{tabularx}{\textwidth}{|l|X|}\hline
        Categoria           & Resultado                                                        \\ \hline
        Frequência          & 3.661 incidentes, sendo (82,8\%) com vazamento confirmado de dados.    \\ \hline
        Atores de ameaça    & (100\%) externos (\textit{breaches}).                                        \\ \hline
        Motivações          & (95\%) financeiras, (5\%) espionagem (\textit{breaches}).                      \\\hline
        Dados comprometidos & Credenciais (50\%), pessoais (41\%), internos (20\%), outros (14\%). \\ \hline
    \end{tabularx}
    \label{quad: quadro1}
    %\vspace{0.5em}
    \caption*{\raggedright\small \textbf{Fonte}: \citeonline{verizon2024dbir}, adaptado.}
    \end{minipage}
    \end{quadro}

    Um exemplo de Tabela.

    \begin{table}[ht]
    \centering
    \begin{minipage}{0.60\textwidth}
    \phantomsection
    \fontsize{10}{12}\selectfont
    \caption{Notificações formais de incidentes e vulnerabilidades em órgãos públicos entre 2021 e 2025.}
    \label{tab:1}
    \begin{tabular}{cccc}
        \toprule
        Ano & Vulnerabilidades & Incidentes & Total de notificações \\
        \midrule
        2025 & 1994 & 4859 & 6853 \\
        2024 & 5115 & 9803 & 14918 \\
        2023 & 10225 & 4905 & 15130 \\
        2022 & 5128 & 3402 & 8530 \\
        2021 & 4964 & 4903 & 9867 \\ \hline
        Total & 27426 & 27872 & 55298 \\
        \bottomrule
    \end{tabular}
        \caption*{\raggedright\small \textbf{Fonte}: \citeonline{ctirgov2025numeros}.}
        \label{quad:cronograma_otimo}
    \end{minipage}
    \end{table}

    \subsubsection{O que escrever na Fundamentação Teórica}

    A Fundamentação Teórica é a base conceitual do seu trabalho. Nela, você apresenta, discute e analisa as teorias, conceitos, modelos e estudos já existentes que sustentam a sua pesquisa.

    Incluir conceitos e definições – Explique os principais termos, conceitos e elementos que serão usados no trabalho, sempre com referência a autores da área.

    Modelos, teorias e abordagens – Traga as principais correntes teóricas que embasam seu estudo.

    Dicas importantes:

        \begin{itemize}
            \item Sempre cite as fontes de onde retirou as informações (seguindo as normas da ABNT).
            \item Organize o texto de forma lógica, por temas ou subtemas, evitando apenas listar autores.
            \item Não copie trechos longos; prefira escrever com suas palavras e citar corretamente.
            \item Evite incluir opiniões pessoais — mantenha o foco no que já foi publicado por outros autores.
        \end{itemize}

        Relacionamento com o seu trabalho – Mostre como essas teorias e estudos se aplicam ou se relacionam com a sua pesquisa.


    \chapter{Trabalhos Relacionados}
    \label{chapter:trabalhos}

    Esta seção apresenta uma revisão de trabalhos relevantes que servem de base para a pesquisa sobre de threads e threads virtuais no java.

    \section{Uma análise comparativa entre threads e green threads no Java}
        \citeonline{Souto2024} realizou uma pesquisa comparando o desempenho de threads tradicionais e threads virtuais, analisando o tempo necessário para instanciá-las, iniciá-las, finalizá-las e realizar a troca de contexto em ambas.

        O autor utilizou a biblioteca JMH do JDK para coletar os tempos e realizar os testes. Para o teste de instanciamento, foram criadas 100.000 threads por meio da chamada new Thread(). Para o teste de inicialização, também foram criadas 100.000 threads, iniciadas com thread.start(). No teste de finalização, 10.000 threads foram armazenadas em uma lista, iniciadas e sincronizadas com thread.join(). Para medir a troca de contexto, foram criadas 100.000 threads, cada uma executando uma pausa de 100 ms e cedendo voluntariamente o controle da CPU por meio de Thread.yield(), simulando operações bloqueantes.

        O estudo demonstrou que, nos cenários testados, as threads virtuais podem ser mais de 100 vezes mais rápidas que as threads tradicionais, evidenciando sua eficiência em operações de criação, inicialização, finalização e mudança de contexto.
    \section{Benchmarking the Performance of Java Virtual Threads in High-Throughput Workloads}
        \citeonline{Pandita2024} apresentou como as threads virtuais funcionam em comparação às threads tradicionais, considerando escalabilidade, utilização de recursos e latência em cargas de trabalho de alto rendimento. O estudo identificou os cenários em que cada modelo se destaca e forneceu dados para auxiliar na tomada de decisões sobre a adoção de Virtual Threads em aplicações Java.

        Foram realizados dois benchmarks: CPU-bound e I/O-bound. No teste CPU-bound, calculou-se a lista de números primos de um limite inferior a um limite superior. No teste I/O-bound, simularam-se operações de bloqueio com duas aplicações Java: uma realiza requisições GET e recebe os resultados da outra aplicação, que introduz um tempo específico de bloqueio antes de produzir a resposta. Todos os benchmarks foram executados em um ambiente controlado na AWS EC2.

        Os resultados mostraram que, para cargas de trabalho com uso limitado de CPU, threads virtuais e threads tradicionais apresentam desempenho semelhante. Já em tarefas com uso intensivo de CPU, surgem gargalos devido aos recursos computacionais, tornando o aumento da simultaneidade pouco relevante. Para cargas de trabalho vinculadas a I/O, as threads virtuais superaram as threads tradicionais em termos de taxa de transferência e latência, permitindo maior simultaneidade, redução do consumo de memória e melhor utilização da CPU.

    \section{Comparison of Concurrency Technologies in Java}
        Em seu trabalho, \citeonline{GustafssonPersson2024} realizou quatro benchmarks com o objetivo de comparar o desempenho entre threads tradicionais, threads virtuais e Reactive Framework em testes de I/O-bound (simulado), CPU-bound e testes mistos.

        Para o teste I/O-bound, foi utilizado o mecanismo Thread.sleep(100 ms) para simular tempos de espera. No teste CPU-bound, realizou-se a multiplicação de matrizes 200x200. Houve ainda dois testes mistos: um focado em I/O-bound, que multiplica matrizes 50x50 e realiza uma pausa de 100 ms, e outro focado em CPU-bound, que multiplica matrizes 200x200 e realiza uma pausa de 50 ms.

        Os resultados mostraram que, no teste CPU-bound, as threads tradicionais apresentaram melhor desempenho em termos de throughput e latência. No teste I/O-bound, o Reactive Framework apresentou maior escalabilidade, menor uso de CPU e memória. Nos testes mistos, a abordagem de threads virtuais obteve alta taxa de requisições e baixa latência no cenário focado em CPU-bound, e manteve desempenho elevado no cenário focado em I/O-bound, enquanto as threads tradicionais apresentaram maior latência e menor throughput.


    \section{Avaliação dos mecanismos de concorrência na API do Java 8}
        Uma análise comparativa do desempenho de single thread, threads, ExecutorService e Fork/Join em algoritmos de ordenação foi realizada por \citeonline{Aguas2015} , utilizando os algoritmos Quicksort, Merge-sort e Pigeonhole Sort.

        Os algoritmos foram testados em três máquinas com processadores diferentes (Dual-Core, i5 e i7). Em todos os casos, cada algoritmo foi executado 60 vezes para cada abordagem de concorrência: single thread, threads, ExecutorService e Fork/Join.

        O Quicksort executado com ForkJoin-Pool em uma máquina com 8 processadores lógicos apresentou o melhor desempenho em termos de tempo. De forma geral, Quicksort combinado com ForkJoin-Pool registrou os menores tempos na maioria dos cenários testados. O estudo demonstrou que a máquina com maior número de processadores lógicos obteve os melhores resultados, comprovando que a execução paralela se beneficia diretamente da disponibilidade de mais núcleos para distribuir as tarefas simultaneamente.
  
    \ignore{
        A seção Trabalhos Relacionados apresenta estudos, projetos ou soluções já desenvolvidos que tratam de problemas semelhantes ou próximos ao do seu trabalho. O objetivo é mostrar o que já foi feito, quais métodos foram usados e onde ainda existem lacunas que justificam sua pesquisa.

        O que incluir:
            \begin{enumerate}
                \item Seleção dos trabalhos – Escolha pesquisas, artigos, relatórios ou projetos relevantes e recentes, de preferência de fontes acadêmicas confiáveis.
                \item Breve descrição – Explique, de forma sucinta, o objetivo, metodologia e resultados de cada trabalho analisado.
                \item Comparação – Mostre similaridades e diferenças entre os trabalhos existentes e o seu.
                \item Identificação de lacunas – Destaque aspectos que não foram explorados ou limitações nas pesquisas anteriores que o seu trabalho pretende abordar.
            \end{enumerate}

        Dicas importantes:

            \begin{itemize}
                \item Organize a apresentação dos trabalhos por tema, abordagem ou cronologia, para manter a lógica do texto.
                \item Utilize citações corretas, conforme a ABNT.
                \item Evite apenas listar trabalhos; faça conexões entre eles e explique a relevância para a sua pesquisa.
                \item Seja objetivo — não é necessário descrever cada detalhe técnico de outros trabalhos, apenas o suficiente para contextualizar sua comparação.
                \item Importante: Ao final dessa seção, o leitor deve entender em que contexto seu trabalho se encaixa e por que ele é necessário, mesmo diante de outras pesquisas já realizadas.
            \end{itemize}
    }

    \chapter{Metodologia ou Procedimentos Metodológicos ou Materiais e Métodos}  
    \label{chapter:metodologia}
  

        A diferença entre Metodologia, Procedimentos Metodológicos e Materiais e Métodos está mais no enfoque e na tradição da área do que em uma mudança drástica de significado. O aluno deve apresentar claramente como o trabalho foi realizado, para que outra pessoa possa entender e, se necessário, reproduzir a pesquisa.
    \section{Metodologia}

        É o termo mais amplo e refere-se ao caminho adotado para realizar a pesquisa.

        Descreve o tipo de pesquisa (exploratória, descritiva, experimental etc.), a abordagem (qualitativa, quantitativa ou mista) e a estratégia geral utilizada.

        Envolve a justificativa das escolhas, por que esse método foi escolhido.

    Exemplo: “Este estudo adotou uma abordagem quantitativa e experimental, utilizando simulações computacionais para avaliar o desempenho de diferentes algoritmos de roteamento.”

    \section{Procedimentos Metodológicos}

        É um termo mais usado nas Ciências Humanas e Sociais. Tem foco no passo a passo da pesquisa: como os dados foram coletados, quais instrumentos foram usados e como a análise foi feita. É mais descritivo e menos técnico do que “Materiais e Métodos”.

    Exemplo: “Para a coleta de dados, foram aplicados questionários estruturados a 50 participantes. As respostas foram analisadas por meio de estatística descritiva e teste t de Student.”

    \section{Materiais e Métodos}

        É muito usado nas Ciências Exatas, Biológicas e Engenharias. Apresenta de forma técnica e detalhada os materiais, ferramentas, softwares, equipamentos ou reagentes utilizados; o passo a passo técnico para realizar o experimento ou implementação.

        Permite que outro pesquisador repita o estudo. 
        
        Exemplo: “O experimento utilizou cinco kits LEGO Mindstorms EV3, computadores com sistema operacional Linux e o software EV3 Classroom. Os testes foram conduzidos em laboratório controlado, com turmas de no máximo dez alunos por sessão.”


    \chapter{Aplicação / Implementação / Experimento}
    \label{chapter:resultados}

    Descrição passo a passo da execução do estudo.
    Apresentação de scripts, fluxogramas, diagramas ou imagens ilustrativas

    \chapter{Conclusão}
    \label{chapter:conclusao}

    A conclusão tem o papel de encerrar o trabalho, retomando de forma resumida o que foi feito e destacando as contribuições obtidas. Ela deve responder à pergunta central da pesquisa e deixar claro o que foi aprendido, comprovado ou desenvolvido.

    A conclusão deve fechar o trabalho com chave de ouro, respondendo à pergunta de pesquisa, destacando o que foi aprendido e mostrando como o estudo contribui para a área, além de abrir portas para novas pesquisas. 

    \begin{itemize}
        \item Relembre brevemente o objetivo geral do estudo e confirme se ele foi atingido.
        \item Destaque os resultados mais importantes, sem repetir tabelas ou gráficos.
        \item Foque no que é mais relevante para responder à questão de pesquisa.
        \item Interprete brevemente o que os resultados significam no contexto do problema.
        \item Falar das limitações do estudo é opcional, mas recomendado. Reconheça possíveis limitações que possam ter influenciado os resultados.
        \item Sugestões para trabalhos futuros. Indique possíveis melhorias ou novas abordagens que podem ser exploradas.
    \end{itemize}

    Não introduza informações novas que não tenham aparecido no desenvolvimento. Use tempo passado para descrever o que foi feito.

    \bookmarksetup{startatroot}% 
    
    ---------------------------------------------------------------------------------------------
    \postextual
    \bibliography{abntex2-modelo-references}

    %\begin{anexosenv}
        %\glossary
        %\begin{apendicesenv}
            %\partapendices
            %\chapter{Quadros de horários}
            %\label{chapter:apendice_quad_hor}
        %\end{apendicesenv}
    %\end{anexosenv}

\end{document}