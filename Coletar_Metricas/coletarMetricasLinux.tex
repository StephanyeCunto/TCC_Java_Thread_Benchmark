\documentclass[12pt,a4paper]{article}
\usepackage{fontspec}
\usepackage{array}
\usepackage{geometry}
\usepackage{graphicx}
\usepackage{xcolor}
\usepackage{setspace}
\usepackage{caption}
\usepackage{booktabs}
\usepackage{titlesec}
\usepackage{colortbl}
\usepackage{float} 
\usepackage{longtable}
\usepackage{colortbl}

\geometry{left=2cm, right=2cm, top=2cm, bottom=2cm}

\setmainfont{Latin Modern Roman}
\setstretch{1.15}

\definecolor{azul}{HTML}{1F4E79}
\definecolor{cinza}{HTML}{F7F7F7}
\definecolor{cinzaA}{HTML}{EDEDED}
\definecolor{cinzaB}{HTML}{EFEFEF}

\titleformat{\section}{\large\bfseries\color{azul}}{}{0pt}{}
\titleformat{\subsection}{\normalsize\bfseries\color{azul}}{}{0pt}{}

\begin{document}

    \begin{center}
        {\LARGE\bfseries Monitoramento de Recursos em Linux}\\[0.3em]
        {\small Comparativo entre ferramentas de análise de desempenho (\texttt{mpstat}, \texttt{vmstat}, \texttt{iostat})}
    \end{center}
    \vspace{1em}

    \section*{Resumo}
    Este documento apresenta um panorama das principais ferramentas de monitoramento de desempenho em sistemas Linux (\texttt{mpstat}, \texttt{vmstat} e \texttt{iostat}), detalhando suas métricas e exemplos de saída.

    \section*{1. Resumo das Ferramentas}

    \begin{center}
        \rowcolors{2}{white}{cinza}
        \renewcommand{\arraystretch}{1.25}
        \begin{tabular}{|p{3cm}|p{6cm}|p{6cm}|}
            \hline
            \rowcolor{azul!15}
            \textbf{Recurso} & \textbf{Métricas típicas} & \textbf{Ferramentas Linux} \\ \hline
            CPU & \%usr, \%nice, \%sys, \%iowait, \%irq, \%soft, \%steal, \%guest, \%gnice, \%idle & \texttt{mpstat} \\ \hline
            Memória RAM & r, b, swpd, free, buff, cache, si, so, bi, bo, in, cs, us, sy, id, wa, st, gu & \texttt{vmstat} \\ \hline
            Disco & Device, r/s, rkB/s, rrqm/s, \%rrqm, r\_await, rareq-sz, w/s, wkB/s, wrqm/s, \%wrqm, w\_await, wareq-sz, d/s, dkB/s, drqm/s, \%drqm, d\_await, dareq-sz, f/s, f\_await, aqu-sz, \%util & \texttt{iostat} \\ \hline
        \end{tabular}
    \end{center}

    \vspace{1em}
    \section*{2. CPU --- \texttt{mpstat}}

    \noindent\textbf{Comando:} \texttt{azureuser@linux-java-vm:\$ mpstat 1 5}

    \vspace{0.8em}
    \begin{center}
        \rowcolors{2}{white}{cinzaA}
        \begin{tabular}{|p{1.3cm}|p{0.7cm}|p{0.7cm}|p{0.85cm}|p{0.8cm}|p{1.3cm}|p{0.8cm}|p{0.8cm}|p{1cm}|p{1.1cm}|p{1.1cm}|p{1cm}|}
            \hline
            \rowcolor{azul!15}
            Hora & CPU & \%usr & \%nice & \%sys & \%iowait & \%irq & \%soft & \%steal & \%guest & \%gnice & \%idle \\ \hline
            20:32:54 & all & 0.00 & 0.00 & 0.25 & 0.00 & 0.00 & 0.00 & 0.00 & 0.00 & 0.00 & 99.75 \\ \hline
            20:32:55 & all & 0.25 & 0.00 & 0.25 & 0.00 & 0.00 & 0.00 & 0.00 & 0.00 & 0.00 & 99.50 \\ \hline
            20:32:56 & all & 0.00 & 0.00 & 0.00 & 0.00 & 0.00 & 0.00 & 0.00 & 0.00 & 0.00 & 100.00 \\ \hline
            20:32:57 & all & 0.25 & 0.00 & 0.50 & 0.00 & 0.00 & 0.00 & 0.00 & 0.00 & 0.00 & 99.25 \\ \hline
            20:32:58 & all & 0.00 & 0.00 & 0.00 & 0.00 & 0.00 & 0.00 & 0.00 & 0.00 & 0.00 & 100.00 \\ \hline
            \textbf{Average} & all & 0.10 & 0.00 & 0.20 & 0.00 & 0.00 & 0.00 & 0.00 & 0.00 & 0.00 & 99.70 \\ \hline
        \end{tabular}
    \end{center}

    \vspace{0.6em}
    \noindent\textbf{Significado das colunas:}
    \begin{center}
        \rowcolors{2}{white}{cinzaA}
        \renewcommand{\arraystretch}{1.2}
        \begin{tabular}{|p{3cm}|p{12cm}|}
            \hline
            \rowcolor{azul!15}
            \textbf{Coluna} & \textbf{Significado detalhado} \\ \hline
            \%usr & Percentual de CPU usado por processos do usuário (aplicações). \\ \hline
            \%nice & Percentual de CPU usado por processos com prioridade ajustada (“nice”). \\ \hline
            \%sys & Percentual de CPU usado pelo kernel. \\ \hline
            \%iowait & Percentual de tempo aguardando I/O. \\ \hline
            \%irq & Percentual usado por interrupções de hardware. \\ \hline
            \%soft & Percentual usado por interrupções de software. \\ \hline
            \%steal & Percentual de CPU “roubado” pelo hypervisor (virtualização). \\ \hline
            \%guest & Percentual usado por VMs convidadas. \\ \hline
            \%gnice & Percentual usado por processos “nice” dentro de VMs convidadas. \\ \hline
            \%idle & Percentual de CPU ociosa. \\ \hline
        \end{tabular}
    \end{center}

    \vspace{1em}
    \section*{3. Memória --- \texttt{vmstat}}

    \noindent\textbf{Comando:} \texttt{azureuser@linux-java-vm:\$ vmstat 1 5}

    \vspace{0.6em}
    \begin{center}
        \scriptsize
        \rowcolors{2}{white}{cinzaA}
        \begin{tabular}{|p{0.25cm}|p{0.25cm}|p{0.9cm}|p{1.6cm}|p{0.9cm}|p{1.15cm}|p{0.25cm}|p{0.25cm}|p{0.45cm}|p{0.45cm}|p{0.45cm}|p{0.55cm}|p{0.25cm}|p{0.25cm}|p{0.45cm}|p{0.35cm}|p{0.35cm}|p{0.35cm}|}
            \hline
            \rowcolor{azul!15}
            \multicolumn{2}{|c|}{\textbf{procs}} &
            \multicolumn{4}{c|}{\textbf{memory}} &
            \multicolumn{2}{c|}{\textbf{swap}} &
            \multicolumn{2}{c|}{\textbf{io}} &
            \multicolumn{2}{c|}{\textbf{system}} &
            \multicolumn{6}{c|}{\textbf{cpu}} \\ \hline
            r & b & swpd & free & buff & cache & si & so & bi & bo & in & cs & us & sy & id & wa & st & gu \\ \hline
            1 & 0 & 0 & 32112212 & 22060 & 341996 & 0 & 0 & 114 & 472 & 139 & 0 & 0 & 0 & 100 & 0 & 0 & 0 \\ \hline
            0 & 0 & 0 & 32112212 & 22060 & 341996 & 0 & 0 & 0 & 0 & 302 & 145 & 0 & 0 & 100 & 0 & 0 & 0 \\ \hline
            0 & 0 & 0 & 32112212 & 22060 & 341996 & 0 & 0 & 0 & 0 & 230 & 131 & 0 & 0 & 100 & 0 & 0 & 0 \\ \hline
            0 & 0 & 0 & 32112212 & 22060 & 341996 & 0 & 0 & 0 & 0 & 161 & 109 & 0 & 0 & 100 & 0 & 0 & 0 \\ \hline
            0 & 0 & 0 & 32112212 & 22060 & 341996 & 0 & 0 & 0 & 4 & 630 & 1087 & 0 & 1 & 99 & 0 & 0 & 0 \\ \hline
        \end{tabular}
    \end{center}

      \vspace{0.6em}
      \begin{center}
        \rowcolors{2}{white}{cinzaA}
        \renewcommand{\arraystretch}{1.2}
        \begin{tabular}{|c|c|p{9cm}|}
            \hline
            \textbf{Categoria} & \textbf{Campo} & \textbf{Descrição} \\ \hline

            {\textbf{procs}} 
            & r & Número de processos prontos para execução (na fila de CPU). Valores altos indicam sobrecarga. \\ \cline{2-3}
            & b & Processos bloqueados aguardando operações de entrada/saída (I/O). \\ \hline

            {\textbf{memory}} 
            & swpd & Quantidade de memória virtual (swap) utilizada, em KB. \\ \cline{2-3}
            & free & Quantidade de memória RAM livre. \\ \cline{2-3}
            & buff & Memória usada para buffers de sistema de arquivos. \\ \cline{2-3}
            & cache & Memória usada como cache de páginas de dados frequentemente acessados. \\ \hline

            {\textbf{swap}} 
            & si & Quantidade de dados lidos do swap para a memória (swap in), em KB/s. \\ \cline{2-3}
            & so & Quantidade de dados gravados da memória para o swap (swap out), em KB/s. \\ \hline

            {\textbf{io}} 
            & bi & Blocos lidos do disco por segundo. \\ \cline{2-3}
            & bo & Blocos gravados no disco por segundo. \\ \hline

            {\textbf{system}} 
            & in & Número de interrupções de hardware por segundo. \\ \cline{2-3}
            & cs & Número de trocas de contexto entre processos por segundo. \\ \hline

            {\textbf{cpu}} 
            & us & Percentual de tempo gasto executando processos de usuário. \\ \cline{2-3}
            & sy & Percentual de tempo gasto em processos do sistema (kernel). \\ \cline{2-3}
            & id & Percentual de tempo ocioso da CPU. \\ \cline{2-3}
            & wa & Percentual de tempo aguardando operações de I/O. \\ \cline{2-3}
            & st & Tempo de CPU “roubado” pelo hypervisor (em ambientes virtualizados). \\ \cline{2-3}
            & gu & Tempo gasto executando máquinas virtuais convidadas (guest). \\ \hline

        \end{tabular}
    \end{center}

    \vspace{1em}
    \section*{4. Disco --- \texttt{iostat}}

    \noindent\textbf{Comando:} \texttt{\$ iostat -x 1 1}

 \vspace{0.6em}
    \begin{center}
    \footnotesize
    \rowcolors{2}{white}{cinzaB}
    \resizebox{\linewidth}{!}{%
        \begin{tabular}{|p{1.2cm}|p{0.6cm}|p{0.9cm}|p{1cm}|p{0.8cm}|p{0.9cm}|p{0.9cm}|
        p{0.7cm}|p{1cm}|p{0.9cm}|p{0.8cm}|p{0.8cm}|p{0.9cm}|
        p{0.8cm}|p{1cm}|p{0.9cm}|p{0.8cm}|p{0.9cm}|p{1cm}|p{0.8cm}|p{0.9cm}|p{0.8cm}|p{0.8cm}|}
            \hline
            \rowcolor{azul!15}
            Device & r/s & rkB/s & rrqm/s & \%rrqm & r\_await & rareq-sz &
            w/s & wkB/s & wrqm/s & \%wrqm & w\_await & wareq-sz &
            d/s & dkB/s & drqm/s & \%drqm & d\_await & dareq-sz &
            f/s & f\_await & aqu-sz & \%util \\ \hline
            loop0 & 0.00 & 0.00 & 0.00 & 0.00 & 0.00 & 0.00 & 0.00 & 0.00 & 0.00 & 0.00 & 0.00 & 0.00 &
            0.00 & 0.00 & 0.00 & 0.00 & 0.00 & 0.00 & 0.00 & 0.00 & 0.00 & 0.00 \\ \hline
            sda & 0.00 & 0.00 & 0.00 & 0.00 & 0.00 & 0.00 & 0.00 & 0.00 & 0.00 & 0.00 & 0.00 & 0.00 &
            0.00 & 0.00 & 0.00 & 0.00 & 0.00 & 0.00 & 0.00 & 0.00 & 0.00 & 0.00 \\ \hline
            sdb & 0.00 & 0.00 & 0.00 & 0.00 & 0.00 & 0.00 & 0.00 & 0.00 & 0.00 & 0.00 & 0.00 & 0.00 &
            0.00 & 0.00 & 0.00 & 0.00 & 0.00 & 0.00 & 0.00 & 0.00 & 0.00 & 0.00 \\ \hline
        \end{tabular}%
    }
    \end{center}

    \vspace{0.6em}
    \noindent\textbf{Significados :}
    \begin{center}
        \rowcolors{2}{white}{cinzaA}
        \renewcommand{\arraystretch}{1.2}
        \begin{tabular}{|c|c|p{9cm}|}
            \hline
            \textbf{Categoria} & \textbf{Campo} & \textbf{Descrição} \\ \hline

            \textbf{Dispositivo} 
            & Device & Nome do dispositivo de armazenamento (ex: \texttt{sda}, \texttt{nvme0n1}). \\ \hline

            \textbf{Leitura (Read)} 
            & r/s & Número de operações de leitura por segundo. \\ \cline{2-3}
            & rkB/s & Quantidade de KB lidos por segundo. \\ \cline{2-3}
            & rrqm/s & Requisições de leitura combinadas por segundo. \\ \cline{2-3}
            & \%rrqm & Percentual de requisições de leitura combinadas. \\ \cline{2-3}
            & r\_await & Tempo médio (ms) de requisições de leitura. \\ \cline{2-3}
            & rareq-sz & Tamanho médio (KB) das requisições de leitura. \\ \hline

            \textbf{Escrita (Write)} 
            & w/s & Operações de escrita por segundo. \\ \cline{2-3}
            & wkB/s & Quantidade de KB escritos por segundo. \\ \cline{2-3}
            & wrqm/s & Requisições de escrita combinadas por segundo. \\ \cline{2-3}
            & \%wrqm & Percentual de requisições de escrita combinadas. \\ \cline{2-3}
            & w\_await & Tempo médio (ms) de escrita. \\ \cline{2-3}
            & wareq-sz & Tamanho médio (KB) das escritas. \\ \hline

            \textbf{Descarte (Discard)} 
            & d/s & Operações de descarte por segundo (TRIM). \\ \cline{2-3}
            & dkB/s & KB descartados por segundo. \\ \cline{2-3}
            & drqm/s & Requisições de descarte combinadas. \\ \cline{2-3}
            & \%drqm & Percentual de descarte combinado. \\ \cline{2-3}
            & d\_await & Tempo médio (ms) de descarte. \\ \cline{2-3}
            & dareq-sz & Tamanho médio (KB) das requisições de descarte. \\ \hline

            \textbf{Flush} 
            & f/s & Operações de flush por segundo. \\ \cline{2-3}
            & f\_await & Tempo médio (ms) para flush. \\ \hline

            \textbf{Fila e Utilização} 
            & aqu-sz & Tamanho médio da fila de requisições pendentes. \\ \cline{2-3}
            & \%util & Percentual de tempo em que o dispositivo esteve ocupado (100\% indica saturação). \\ \hline

        \end{tabular}
    \end{center}

    \vspace{1em}
    \section*{5. Comparativo Geral de Métricas}

    \rowcolors{2}{white}{cinzaA}
    \renewcommand{\arraystretch}{1.3}

    \begin{longtable}{|p{4cm}|p{3.5cm}|p{3.5cm}|p{3.5cm}|}
        \hline
        \rowcolor{azul!15}
        \textbf{Métrica / Conceito} & \textbf{mpstat} & \textbf{vmstat} & \textbf{iostat} \\ \hline
        \endfirsthead

        \hline
        \rowcolor{azul!15}
        \textbf{Métrica / Conceito} & \textbf{mpstat} & \textbf{vmstat} & \textbf{iostat} \\ \hline
        \endhead

        Uso de CPU em modo usuário & \%usr & us & \%user \\ \hline
        Uso de CPU em modo sistema (kernel) & \%sys & sy & \%system \\ \hline
        Tempo ocioso da CPU & \%idle & id & \%idle \\ \hline
        Tempo de espera por I/O & \%iowait & wa & \%iowait \\ \hline
        Tempo gasto com interrupções de hardware & \%irq & -- & -- \\ \hline
        Tempo gasto com interrupções de software & \%soft & -- & -- \\ \hline
        Tempo roubado pelo hypervisor (virtualização) & \%st & st & -- \\ \hline
        Tempo de CPU em máquinas virtuais convidadas & \%guest & gu & -- \\ \hline

        Processos em execução & -- & r & -- \\ \hline
        Processos bloqueados (I/O) & -- & b & -- \\ \hline

        Memória livre & -- & free & -- \\ \hline
        Memória em buffers & -- & buff & -- \\ \hline
        Memória em cache & -- & cache & -- \\ \hline
        Memória de swap utilizada & -- & swpd & -- \\ \hline
        Leitura do swap (swap in) & -- & si & -- \\ \hline
        Escrita no swap (swap out) & -- & so & -- \\ \hline

        Blocos lidos do disco & -- & bi & r/s \\ \hline
        Blocos gravados no disco & -- & bo & w/s \\ \hline
        KB lidos por segundo & -- & -- & rkB/s \\ \hline
        KB escritos por segundo & -- & -- & wkB/s \\ \hline
        Requisições de leitura combinadas & -- & -- & rrqm/s \\ \hline
        Requisições de escrita combinadas & -- & -- & wrqm/s \\ \hline
        Tempo médio de leitura (ms) & -- & -- & r\_await \\ \hline
        Tempo médio de escrita (ms) & -- & -- & w\_await \\ \hline
        Tamanho médio da fila de requisições & -- & -- & aqu-sz \\ \hline
        Percentual de utilização do dispositivo & -- & -- & \%util \\ \hline

        Interrupções por segundo & -- & in & -- \\ \hline
        Trocas de contexto por segundo & -- & cs & -- \\ \hline

    \end{longtable}


\end{document}
