\documentclass[10pt,a4paper]{article}
\usepackage{fontspec}
\usepackage{array}
\usepackage{geometry}
\usepackage{graphicx}
\usepackage{xcolor}
\usepackage{setspace}
\usepackage{caption}
\usepackage{booktabs}
\usepackage{titlesec}
\usepackage{colortbl}
\usepackage{float} 
\usepackage{longtable}
\usepackage{colortbl}

\geometry{left=2cm, right=2cm, top=2cm, bottom=2cm}


\begin{document}

    \section{Métricas de Desempenho: Tempo e Vazão}

    \subsection{Tempo de Execução}

    As medições de tempo são utilizadas para avaliar a eficiência de execução das tarefas em diferentes modos de medição. O JMH fornece diversas métricas para mensurar o tempo gasto em cada operação.

    \begin{itemize}
    \item \textbf{Métrica:} \texttt{AverageTime}, \texttt{SampleTime}, \texttt{SingleShotTime}
    \item \textbf{Unidade:} nanossegundos (ns), microssegundos (\textmu s) ou milissegundos (ms)
    \end{itemize}

    \textbf{Exemplo de saída:}
    \begin{verbatim}
        Benchmark                 Mode  Cnt  Score   Error  Units
        TesteThreadsTradicionais  avgt   30  1.234 ± 0.012  ms/op
    \end{verbatim}

    \noindent\textbf{Interpretação:}
    \begin{itemize}
    \item \textbf{Mode:} Tipo de métrica usada (neste caso, AverageTime)
    \item \textbf{Cnt:} Número de medições válidas (30 repetições)
    \item \textbf{Score:} Tempo médio por operação
    \item \textbf{Error:} Desvio padrão do resultado
    \item \textbf{Units:} Unidade de tempo utilizada
    \end{itemize}


    \vspace{1em}
    \noindent\textbf{Resumo:}
    \begin{center}
    \begin{tabular}{|p{4cm}|p{10cm}|}
    \hline
    \textbf{Métrica} & \textbf{Descrição} \\ \hline
    \texttt{AverageTime} & Tempo médio por operação \\ \hline
    \texttt{SampleTime} & Distribuição estatística dos tempos individuais \\ \hline
    \texttt{SingleShotTime} & Tempo total de uma única execução \\ \hline
    \end{tabular}
    \end{center}

    \subsection{Vazão (Throughput)}

    A métrica \texttt{Throughput} mede o número de operações executadas por segundo, indicando a capacidade do sistema de processar tarefas em determinado intervalo de tempo.

    \begin{itemize}
    \item \textbf{Métrica:} \texttt{Throughput}
    \item \textbf{Unidade:} \texttt{ops/s} (operações por segundo)
    \end{itemize}

    \textbf{Exemplo de saída:}
    \begin{verbatim}
        Benchmark                 Mode  Cnt   Score   Error  Units
        TesteThreadsVirtuais      thrpt 30  8123.45 ± 23.8  ops/s
    \end{verbatim}

    \subsection{CPU e Memória}

    Para analisar o uso de CPU e memória durante a execução dos benchmarks, o JMH permite o uso de \textit{profilers}, ferramentas que coletam informações detalhadas sobre o comportamento da JVM e do sistema.

    \vspace{0.5em}
    \begin{itemize}
        \item \texttt{-prof jfr}: utiliza o \textit{Java Flight Recorder} (JFR) para gerar um arquivo detalhado com métricas de CPU, heap, threads e eventos da JVM.
    \end{itemize}

    \vspace{0.5em}
    \noindent\textbf{Exemplo de uso:}
    \begin{verbatim}
    java -jar target/benchmarks.jar org.sample.TesteProfiler -prof jfr
    \end{verbatim}


\end{document}